%%%%%%%%%%%%%%%%%%%%%%%%%%%%%%%%%%%%%%%%%
% Adarsh Patil CV
% LaTeX Template
% Version 1.0 (24/3/13)
%
%%%%%%%%%%%%%%%%%%%%%%%%%%%%%%%%%%%%%%%%%

% !TeX program = xelatex
%----------------------------------------------------------------------------------------
%	PACKAGES AND OTHER DOCUMENT CONFIGURATIONS
%----------------------------------------------------------------------------------------

\documentclass[a4paper,10pt]{article} % Default font size and paper size

\usepackage{fontspec} % For loading fonts
\defaultfontfeatures{Mapping=tex-text}
\setmainfont[SmallCapsFont = Fontin SmallCaps]{Fontin} % Main document font

\usepackage{xltxtra,xunicode,url,parskip} % Formatting packages

\usepackage[usenames,dvipsnames]{xcolor} % Required for specifying custom colors

\usepackage[big]{layaureo} % Margin formatting of the A4 page, an alternative to layaureo can be \usepackage{fullpage}
% To reduce the height of the top margin uncomment: \addtolength{\voffset}{-1.3cm}

\usepackage{hyperref} % Required for adding links	and customizing them
\definecolor{linkcolour}{rgb}{0,0.2,0.6} % Link color
\hypersetup{colorlinks,breaklinks,urlcolor=linkcolour,linkcolor=linkcolour} % Set link colors throughout the document

\usepackage{titlesec} % Used to customize the \section command
\titleformat{\section}{\Large\scshape\raggedright}{}{0em}{}[\titlerule] % Text formatting of sections
\titlespacing{\section}{0pt}{3pt}{3pt} % Spacing around sections

\newcommand{\tabitem}{~~\llap{\textbullet}~~}

\begin{document}

\pagestyle{empty} % Removes page numbering

%\font\fb=''[cmr10]'' % Change the font of the \LaTeX command under the skills section

%----------------------------------------------------------------------------------------
%	NAME AND CONTACT INFORMATION
%----------------------------------------------------------------------------------------

\par{\centering{\Huge Adarsh \textsc{Patil}}\bigskip\par} % Your name

\section{Personal Details}

\begin{tabular}{rl}
\textsc{Portfolio / Blog:} & \href{https://adarshpatil.in/timewarp}{https://adarshpatil.in} \hspace{2cm} \textsc{Address:} Edinburgh, UK\\
\textsc{email:} & \href{mailto:me@adarshpatil.in}{me@adarshpatil.in}, \href{mailto:adarsh.patil@ed.ac.uk}{adarsh.patil@ed.ac.uk}, \href{mailto:adarsh@iisc.ac.in}{adarsh@iisc.ac.in}\\
\end{tabular}

%----------------------------------------------------------------------------------------
%	EDUCATION
%----------------------------------------------------------------------------------------

\section{Education}

\begin{tabular}{rl}
\textsc{Current}  & \textsc{Doctor of Philosophy} \\
& \textbf{University of Edinburgh}, United Kingdom \qquad \small{[ARM PhD Fellowship]} \\
& Research: \small\emph{Co-designing reliability and performance for datacenter memory} \\
&\\


%------------------------------------------------

\textsc{July} 2017 & \textsc{M.Tech. (Research)} \\
& \textbf{Indian Institute Of Science}, Bangalore, India\\
& Thesis: \small\emph{Heterogeneity Aware Shared DRAM Cache for Integrated Heterogeneous Architectures} \\
&\normalsize \textsc{Gpa}: 6.33/8.0 - \small\emph{magna cum laude}\\
%&\normalsize \textsc{Gpa}: 6.33/8.0 - \small\emph{magna cum laude} \hyperlink{iisc}{\hfill | \footnotesize List of Courses}\\
&\\

%------------------------------------------------

\textsc{May} 2012 & \textsc{Bachelor of Engineering}\\ 
& \textbf{M S Ramaiah Institute of Technology}, Bangalore, India\\
%&\normalsize \textsc{Gpa}: 9.40/10.0 - \small\emph{summa cum laude} \hyperlink{msrit}{\hfill| \footnotesize List of Courses}\\
&\normalsize \textsc{Gpa}: 9.40/10.0 - \small\emph{summa cum laude}\\
&\\

%------------------------------------------------
\iffalse
\textsc{June} 2008 & \textbf{Sindhi High School}, Hebbal, Bangalore, India\\
& \textsc{CBSE 12}\textsuperscript{th}, \normalsize \textsc{Percentage}: 87.56 (PCM: 93) \hyperlink{hs}{\hfill| \footnotesize List of Courses}\\
&\\

%------------------------------------------------

\textsc{July} 2006 & \textbf{Presidency School}, R T Nagar, Bangalore, India\\
& \textsc{ICSE 10}\textsuperscript{th}, \normalsize \textsc{Percentage}: 86.5 \\
\fi

\end{tabular}

%----------------------------------------------------------------------------------------
%	WORK EXPERIENCE 
%----------------------------------------------------------------------------------------

\section{Work Experience $\sim$ 4 years}

\begin{tabular}{r|p{10cm}}
Aug 2017 - Apr 2019 & Research Scientist at \textsc{Intel Corporation}, Bangalore, India \\
{\footnotesize(1 year 8 months)} & \footnotesize{HPC Ecosystem and Application Team} - projects under NDA *\\
&\\
%------------------------------------------------

Jun 2012 - Jul 2014 & Technology Analyst at \textsc{Goldman Sachs}, Bangalore, India \\
{\footnotesize (2 years 1 month)} & \footnotesize{Core Platform Engineering}\\ 
& \\
%------------------------------------------------

%Jan 2012 - May 2012 & Intern at \textsc{Ignis Technology Solutions}, Bangalore, India \\
%& \footnotesize{Android app developer}\\
%& \\

%------------------------------------------------

Jun 2011 - Aug 2011 & Summer Analyst at \textsc{Goldman Sachs}, Bangalore, India \\
& \footnotesize{Runbook Process Automation}\\
& \\

%------------------------------------------------

Jun 2010 - Sep 2010 & Intern at \textsc{Gavista Tech}, Bangalore, India\\
& \footnotesize{Search Optimization and User Interface for systematic results display}\\

\end{tabular}


%----------------------------------------------------------------------------------------
%	PUBLICATIONS 
%----------------------------------------------------------------------------------------
\vspace{-0.1em}
\section{Publications \& Talks}
\begin{tabular}{p{3cm}p{11cm}}
DSN 2023 &  $\bar{A}$pta: Fault-tolerant object-granular CXL disaggregated memory for \newline accelerating FaaS \\
& \\
ISCA 2021 & Dv\'e: Improving DRAM Reliability and Performance On-Demand via \newline Coherent Replication \hfill  \href{https://adar.sh/dve}{https://adar.sh/dve}\\
&\\
TACO 2017 \newline \footnotesize{Best Poster EECS 2017, Presented HiPEAC 2018} & HAShCache: Heterogeneity-Aware Shared DRAMCache for Integrated Heterogeneous Systems  \hfill
\href{https://adar.sh/hashcache-taco}{https://adar.sh/hashcache-taco}\\
&\\
ARM/UEd Conf 2021 \footnotesize{(Talk)} & Improving Reliability and Performance of Datacenter Systems via \newline Coherence \hfill  \href{https://adar.sh/arm-ed-conf-2021}{https://adar.sh/arm-ed-conf-2021}\\	
&\\
UK Systems 2021 \footnotesize{(Talk)} & FaaS with CXL Disaggregated Shared Memory \\
\end{tabular}
%----------------------------------------------------------------------------------------
%	PROJECTS 
%----------------------------------------------------------------------------------------

\section{Projects}

\begin{tabular}{rp{12cm}}
\textsc{Ongoing Projects} & Redesigning datacenter co-ordination services for next generation CXL based shared memory \\
%& Next generation CXL based memory disaggregation brings performance and availability as a hardware primitive. This makes \\
& \\
& Achieving persistence through replicated disaggregated memory \\
& \\
\textsc{Projects at} & TLB and Pagewalk Performance in Multicore Architectures with large \\
\textsc{INDIAN INSTITUTE} & Die-Stacked DRAM Cache \hfill [Tech Report 2015, arXiv]\\
\textsc{OF SCIENCE} & ~~~\href{https://adar.sh/caffe-compiler-optimize}{https://adar.sh/caffe-compiler-optimize}\\
& \\
& A Study of Branch Prediction in Android \\
& ~~~\href{https://adar.sh/BranchPredAndroid}{ https://adar.sh/BranchPredAndroid} \\
& \\
& Compiler Optimization Transforms\\
& ~~~Harris Corner Detection: \href{https://adar.sh/compiler-optimize}{https://adar.sh/compiler-optimize}\\
& ~~~Caffe Neural Networks:  \href{https://adar.sh/caffe-compiler-optimize}{https://adar.sh/caffe-compiler-optimize}\\
& \\
& VarMutate: Dynamic Scoping for C Language in clang\\
& ~~~\href{https://adar.sh/VarMutate}{https://adar.sh/VarMutate} \\
& \\
& Plan IKEBANA: ESS Dimensions Reduction for Plan Bouquet \\
& ~~~\href{https://adar.sh/PlanIkebana}{https://adar.sh/PlanIkebana}\\
&\\
\textsc{Projects at}  &  Architect, design and implement solutions of various virtualization \\
\textsc{GOLDMAN SACHS} &   \& linux technologies spanning datacenter compute, storage, networking \\
&\\
& Hardware and OS Performance Benchmarking \& Analysis\\
& \tabitem Authored an automated benchmarking framework to run and \\
& ~~~~ report performance by running test suites on VMs and Baremetals\\
& \tabitem Performance analysis \& tuning for specialized internal apps \\
& ~~~( e.g. low Latency, high I/O, memory, network intensive )\\
& \tabitem Test Suites – SpecJBB, kmake, blacksholes, Dhrystone, \\
& ~~~~Whetstone, Hackbench, Disk tests, Network uperf, lat proc \\

&\\
& Linux Containers\\
& \tabitem Architecting and implementing Containers for Goldman Sachs Cloud \\
& \tabitem Possess a good understanding of underlying technology \\
& ~~~~Namespaces, Cgroups, SELinux, Network configuration, Libvirt API \\
&\\
& Thin client desktop VDI solution\\
& \tabitem Engineered a Minimized and locked down Linux based solution \\
& \tabitem Authored several PyGTK and X11 based applications for remote \\
& ~~~~management, diagnostics, troubleshooting and NEA \\
& \tabitem Network booted, kickstart and preseed based unsupervised install \\
& \tabitem Engineered a stateless RAM-based network booted system on\\ & ~~~~ARM based hardware\\
&\\
& Engineering Nested Virtualization (Bromium vSentry) as a security solution\\
& \\
& Vendor Interaction and liaising – Intel, VMware, Redhat \\
&\\
\textsc{Projects at} & Spoken language identification using machine learning \\ 
\textsc{M S Ramaiah} & ~~~ [Bachelor's dissertation] \qquad \qquad \href{https://adar.sh/spokenlang}{https://adar.sh/spokenlang} \\
\textsc{Inst of Tech} & \\
& SNIDS: An Intelligent \& Multiclass Support Vector Machines Based NIDS \\ 
& ~~~ [ICECIT 2012] \qquad \qquad \qquad \qquad \quad \href{https://adar.sh/S-NIDS}{https://adar.sh/S-NIDS} \\
& ~~~ funded by Defense Research and Development Organization (DRDO), India \\
& \\
& Line Birds (game) using OpenGL\qquad \href{https://adar.sh/linebird}{https://adar.sh/linebird}\\
& A parallel algorithm for Max Flow Algorithm using Ford-Fulkerson method \\
& Lead developer of a student focused Linux Distro ``ANDROMEDA Linux" \\
%& e-Blood Bank - a scalable database application\\
\end{tabular}

\iffalse
%----------------------------------------------------------------------------------------
%	SOFTWARE and Hardware proficiency
%----------------------------------------------------------------------------------------

\section{Software \& Hardware Proficiency}
\begin{tabular}{rl}
	\textsc{Software} & \tabitem \textit{Architectural Simulators:}\\
	& ~~~~gem5-gpu, gem5, DRAMSim, MARSSx86(QEMU based fast-functional simulator),\\
	& \tabitem \textit{Programming Languages:}\\
	& ~~~~C / C++, Python, Linux Shell Scripting, Java, SQL, Perl\\
	& \tabitem \textit{Development Platforms:} \\
	& ~~~~VI / VIM, Visual Studio Code, gedit, TexMaker\\
	& \tabitem \textit{Web Technology:} \\
	& ~~~~HTML, CSS, JQuery, Javascript, PHP, JSP, Curl, XML, REST, WSDL, Perl / Python CGI\\
	& \tabitem \textit{Operating Systems:} VMware ESX, Linux(RHEL, Fedora, Ubuntu), Windows \\
	& \tabitem \textit{Mobile Platforms:} Android \\
	& \tabitem \textit{Libraries:} CUDA, OpenCL, OpenCV, OpenGL  \\
	& \tabitem \textit{Tools:} \LaTeX\ , git, svn, hg, gdb\\
	& \tabitem \textit{Automation Tool:} HP Operations Orchestrator, iConclude \\
	& \tabitem \textit{Database:} MySQL, Oracle, DB2 \\
	& \tabitem \textit{Browser Based Applications:} Chrome apps and XULRunner \\

	&\\

	\textsc{Hardware}  & \tabitem Hardware Accelerators \\
	& ~~~~Intel MIC (XeonPhi KNC) and NVIDIA GPGPUs (Fermi, Kepler) \\
	& \tabitem Micro-Architecture of Modern x86 CPUs\\
	& ~~~~(Nehalem / Westmere, SandyBridge / IvyBridge, Haswell) \\
	& \tabitem 8051 based Microcontroller Programming \\
	&\\
\end{tabular}
\fi

%----------------------------------------------------------------------------------------
% Achievements
%----------------------------------------------------------------------------------------

\section{Achievements and awards}
~\\
\tabitem Founding trustee of Dr. M R Gorbal Foundation - a charitable organization which aims to promote research in Physics (2022) \\
~\\
\tabitem Best Poster at Electrical Science Divisional Symposium at IISc, Bangalore (2017)\\
\\
\tabitem Completed with certificate of distinction several Data Science Courses from Johns Hopkins University on Coursera (2014)\\
\\
\tabitem \textit{``Best outgoing achiever (2012)”} - Dept. of CSE at M S Ramaiah Instiute of Technology \\
\\
\tabitem First Place at National Level Project Competition held at M S Ramaiah Inst. of Tech (2012)\\
\\
\tabitem Second Place at “Random Hacks of Kindness \#2” hackathon (2010)\\
%\\
%\tabitem IBM Certified DB2 9 Database and Application Fundamentals (2011)\\
%\tabitem IBM Certified Rational Functional Tester(RFT) for Java (2011)\\
%\tabitem Certificate course in “Java Programming” from NIIT-Bangalore by Sun Microsystems (2009)\\
%\\
%\tabitem Study titled ``Microneedles for medical advancement" (MEMS course project) - recognized by the Staff Council of IEEE MSRIT.\\
%\tabitem Credited broad course electives like “Micro-Electro Mechanical Systems (MEMS)”, \\“Digital Signal Processing” and “Supply Chain Management” at MSRIT \\
%----------------------------------------------------------------------------------------

%----------------------------------------------------------------------------------------
%	POSITIONS HELD
%----------------------------------------------------------------------------------------

\section{Voluntary Positions Held}
\hskip-0.4cm
\begin{tabular}{ll}
	\tabitem Informatics Science Communication Group & Dec 2021 - Current \\
	&\\
	\tabitem ICSA@Informatics social media communication & Sept 2021 - Current \\
	& \\
	\tabitem Teaching assistant/Tutor INF2C-CS, University of Edinburgh & Aug 2019 - Dec 2019 \\
	& \\
	\tabitem Student System Admin at CSA Department, IISc & Aug 2014 - Dec 2016 \\
	&\\
	\tabitem Teaching Associate for the CUDA Teaching Centre,  & Jan 2012 - May 2012\\
	~~~~sponsored by NVIDIA, at the Department of CSE, MSRIT &  \\
	&\\
	\tabitem Chairman of VRGLINUG (GNU/Linux users group at MSRIT) & 2011-12 \\
	~~~ Secretary and member of executive committee of IEEE-MSRIT &  \\
	~~~ Influential Member of several committees & \\
	~~~ {\footnotesize (RoboMSR, CodeMSRIT, Assoc. of Computer Engineers)} & \\
	%	&\\
\end{tabular}

%----------------------------------------------------------------------------------------
%	Co and extra curricular activities
%----------------------------------------------------------------------------------------

\section{Extra curricular activities}
~\\
\tabitem Represented IISc in the 12 hour Bengaluru Stadium Run Relay for 2016 and 2018. \\ Long distance runs : 4 villages half-marathon (UK), TCS World 10k, Bengaluru 10k Challenge, Standard Chartered Mumbai Marathon, Bengaluru Marathon etc. \\ {\footnotesize Personal best times - \textit{5k (19mins), \space 10k (45mins), half-marathon (1:50hrs)}, marathon (4:17hrs) }\\
\\
\tabitem Periodically author blog articles about my experiences, assessments and outlooks related to my work and hobbies \\
\\
\tabitem Member of environment committee at Goldman Sachs for conducting awareness drives/camps \\
\\
\tabitem One of 4 Indians amongst participants from around the world on a Scholarship Delegation to attend TEDxSummit 2012 in Doha, Qatar \\
\\
\tabitem TEDx licensee, TED Translator, TEDx organizer (TEDxMSRIT 2012)\\
%\tabitem Web Design, Development and treasurer for Samanway 2014 - Career fair at IISc \\
\tabitem Organizer of ``Pycon India 2010" and ``Random Hacks of Kindness \#4"\\
%~\tabitem Lead organiser of “Ignite” (annual Tech Fest at MSRIT) and “Aavishkaar” (annual inter-collegiate IEEE-MSRIT tech fest\\\\
%\tabitem Active Volunteer for Association of Computer Engineers (ACE), IEEE-MSRIT and have been instrumental in organizing several fests and events (2008-2011)\\
\tabitem Delegated at various conferences/workshops - ISCA 2022, IPDPS 2015, Open Hack India 2010/2011, PYCON India 2010, FOSSEE 2011, Microsoft DreamSpark 2012, Wikipedia Meetup/Bangalore\\
%\\

%----------------------------------------------------------------------------------------

%----------------------------------------------------------------------------------------
%	Miscellaneous
%----------------------------------------------------------------------------------------

\section{Miscellaneous}
\begin{tabular}{rl}
\textsc{Strengths} & \tabitem Adaptability, Quick learner, Hardworking and Dedication\\
& \tabitem Effective communicator and good leadership skills \\
& \tabitem Always updated with latest technology and trends of market.\\
& \tabitem Analytical and mathematical problem solving ability \\
& \\
\textsc{Hobbies} & \tabitem Avid Runner, Cyclist and Swimmer\\
& \tabitem Cardio/HiiT workouts - Les Mills Body Pump, Body Attack , GRIT Athletic\\
& \tabitem Hiking enthusiast - 5 Munros, several Corbetts, coastal and trail walks\\
& \\
\textsc{Other Links} & \href{https://github.com/adarshpatil}{github.com/adarshpatil} \\
& \href{https://in.linkedin.com/in/adarshpatil}{in.linkedin.com/in/adarshpatil}\\
&\\
\textsc{References} & \textbf{Academic References} \\
& Vijay Nagarajan, PhD Advisor \\
& Professor, University of Edinburgh\\
& vijay.nagarajan@ed.ac.uk \\
&\\
& Prof. R Govindarajan, Master's Advisor \\
& Professor, IISc\\
& govind@serc.iisc.ernet.in \\
&\\

& \textbf{Industry References} \\
& Bharat Kaul \\
& Director, Intel Parallel Computing Lab \\
\end{tabular}

\vfill

\centerline{Created with Xe\LaTeX\ }
%----------------------------------------------------------------------------------------

\iffalse

\newpage

%----------------------------------------------------------------------------------------
%	GRADE TABLES
%----------------------------------------------------------------------------------------

\par{\centering\Large \hypertarget{iisc}{Master of Science in  Engineering (IISc, Bangalore)}\par}\large{\centering Grades\par}\normalsize

\begin{center}
\begin{tabular}{lcc}
\multicolumn{1}{c}{\textsc{Course}} & \textsc{Grade}&\textsc{Credit}\\ \hline
Database Management Systems & A & 4\\
Computer Architecture & A & 4\\
Design and Analysis of Algorithms & C & 4\\
Compiler Design (NOT IN RTP) & B & -\\
Final Thesis &  & Completed \\
&&\\
& Total & 16\\\cline{2-3}
&\textsc{Gpa}&\textbf{6.33}
\end{tabular}
\end{center}
\bigskip
\hrule
\bigskip

%------------------------------------------------

\bigskip

\par{\centering\Large \hypertarget{msrit}{Bachelors in Engineering (M S Ramaiah Inst. of Tech, Bangalore) }\par}
\large{\centering Principal Courses\par}
\normalsize
\begin{center}
\begin{tabular}{ll}
\tabitem Engineering Mathematics & \tabitem Discrete Mathematics \\
\tabitem Data Structures & \tabitem Design \& Analysis of Algorithms \\
\tabitem Operating Systems & \tabitem Computer Organization \\
\tabitem Engineering Design & \tabitem Computer Graphics and Visualization \\
\tabitem Web Programming & \tabitem Advanced Computer Architecture \\
\tabitem Unix System Programming & \tabitem Computer Networks \\
\tabitem Compiler Design & \tabitem Software Engineering \\
&\\
\multicolumn{2}{c}{\large{\centering Electives\par}} \\ 
&\\
\tabitem Artificial Intelligence  & \tabitem Supply Chain Management  \\
\tabitem Digital Signal Processing & \tabitem Micro-Electro Mechanical Systems\\
\end{tabular}
\end{center}
\bigskip
\hrule
\bigskip
%------------------------------------------------

\bigskip


\par{\centering\Large \hypertarget{hs}{Higher Secondary (Sindhi High School, CBSE) }\par}
\begin{center}
\large{Primary Courses\par}
\normalsize
Physics, Chemistry, Mathematics, Computer Science \\
~\\
\large{Languages\par}
\normalsize
Hindi, English \\
\end{center}
\bigskip
\hrule
\bigskip
%----------------------------------------------------------------------------------------
\vfill
\centerline{Created with Xe\LaTeX\ }
\fi
\end{document}
